\documentclass[spanish, fleqn]{article}
\usepackage{babel}
\usepackage[utf8]{inputenc}
\usepackage{fourier}
\usepackage{icomma}
\usepackage{amsmath, amsfonts, amsthm, fourier}
\usepackage[colorlinks, urlcolor=blue]{hyperref}
\usepackage{graphicx}
\usepackage{listings}

\newcommand{\num}{1}
\title{Tarea \num\\
       \large Algoritmos y Complejidad\\[3ex]
       \emph{``Tonteando\ldots''}}
\author{Algorithm Knaves}
\date{2018-03-26}

\begin{document}

\maketitle

  Es común querer saber el valor de la función inversa:
  \begin{align*}
    y
      &= f(x) \\
    x
      &= f^{-1}(y)
  \end{align*}
  Sabemos que \(f^{-1}\) no siempre está definida.
  Para los efectos presentes,
  suponga que existe.

  El problema se hace entretenido si \(f\)
  solo se conoce en puntos discretos\ldots
  Una manera de aproximar \(f^{-1}(y^*)\)
  es interpolar \(f\) para valores adecuados de \(x\)
  y resolver la ecuación \(f(x^*) = y^*\)
  usando un método numérico.
  \begin{enumerate}
  \item % 20181t1p1
    Considere los valores del cuadro~\ref{tab:20181t1p1t1}.
    \begin{table}[ht]
      \centering
      \begin{tabular}{rr}
        \multicolumn{1}{c}{\(\mathbf{x}\)} &
           \multicolumn{1}{c}{\(\mathbf{y}\)} \\
        \hline
        -1,000 & 0,038 \\
        -0,600 & 0,100 \\
        -0,467 & 0,155 \\
        -0,200 & 0,500 \\
        -0,067 & 0,900
      \end{tabular}
      \caption{Tabla de valores de \(f\)}
      \label{tab:20181t1p1t1}
    \end{table}
    Escriba una función en Python
    que toma arreglos \(X\) e \(Y\) de los valores dados,
    y un valor de \(x\)
    y obtiene el valor interpolado de \(y\) correspondiente.
  \item % 20181t1p2
    Escriba una función Python
    que implementa el método de la secante,
    dada la función \(f\), puntos iniciales \(x_0\) y \(x_1\)
    y una tolerancia.
  \item % 20181t1p3
    Use las anteriores para aproximar \(f^{-1}(0,3)\).
  \item % 20181t1p4
    Otra forma de hacerlo es considerar \(x\) una función de \(y\),
    usando interpolación inversa.
    Use sus funciones con esta idea para aproximar \(f^{-1}(0,3)\).
  \end{enumerate}

% Condiciones generales de tareas de Estructuras Discretas, 2015
\section{Condiciones de entrega}

  \begin{itemize}
  \item
    La tarea se realizará \emph{individualmente}
    (esto es grupos de una persona),
    sin excepciones.
  \item
    La entrega debe realizarse vía \href{http://moodle.inf.utfsm.cl}{Moodle}
    en un \emph{tarball} en el área designada al efecto, bajo el formato
    \texttt{tarea-\num-\emph{rol}.tar.gz}
    (\texttt{rol} con dígito verificador y sin guión).

    Dicho \emph{tarball} debe contener las fuentes en LaTeX
    (al menos \texttt{tarea.tex})
    de la parte escrita de su entrega,
    además de un archivo \texttt{tarea-\num.pdf},
    correspondiente a la compilación de esas fuentes.
  \item
    En  caso de haber programas,
    su ejecutable \emph{debe} llamarse \texttt{tarea-\num},
    de haber varias preguntas solicitando programas,
    estos deben llamarse \texttt{tarea-\num-1},
    \texttt{tarea-\num-2},
    etc.
    Si hay programas compilados,
    incluya una \texttt{Makefile}
    que efectúe las compilaciones correspondientes.

    Los programas se evalúan según que tan claros
    (bien escritos)
    son,
    si se compilan y ejecutan sin errores o advertencias según corresponda.
    Parte del puntaje es por ejecución correcta con casos de prueba.
    Si el programa no se ciñe a los requerimientos de entrada y salida,
    la nota respectiva es cero.
  \item
    Además de esto,
    la parte escrita de la tarea debe en hojas de tamaño carta
    en Secretaría Docente de Informática (Piso 1, edificio F3).
  \item
    Tanto el \emph{tarball} como la entrega física
    deben realizarse el día indicado
    en \href{http://moodle.inf.utfsm.cl}{Moodle}.
    No entregar la parte escrita en papel
    o no entregar en formato electrónico tiene un descuento de 50 puntos.

    Por cada día de atraso se descontarán 20 puntos.
    A partir del tercer día de atraso
    no se reciben más tareas,
    se entiende la tarea como no entregada.
  \item
    Nos reservamos el derecho de llamar a interrogación
    sobre algunas de las tareas entregadas.
    En tal caso,
    la nota base
    (antes de descuentos por atraso y otros)
    es la de la interrogación.
    No presentarse a la interrogación sin justificación previa
    significa automáticamente nota cero.
  \end{itemize}


\end{document}

%%% Local Variables:
%%% mode: latex
%%% TeX-master: t
%%% End:
